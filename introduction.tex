\chapter{Introduction}
Online communities have become very popular in the recent times. They have become incredibly helpful in sharing knowledge, ideas, skills or even experiences nowadays. People ask questions on different topics and others who have knowledge on that topic reply to those questions. Thus a knowledge sharing culture has already grown and it's getting a proper shape slowly. A number of QA sites and we based discussion forum are being widely used by millions of users across the world. So it's important to make sure that the knowledge base is not corrupted. In this paper we will introduce major fradulant activities which are frequently found in online communities and also propose different approaches to classify those fradulant activities properly. 


\section{Motiovation}
There is no doubt that the online communities are playing a vital role in learning new things and sharing knowledge. But there is no guaranty that the answer someone has given is absolutely correct. There must be strong logic and references to the answer to be a good one. Thus, awarding some points and other rewards like badges, medals and other cool stuffs to the answerer has become a popular mean to encourage to provide more quality answers. This has somewhat established a trend that the more points and other rewards a person has, the more he knows. 
\\
One of the most widely used 

